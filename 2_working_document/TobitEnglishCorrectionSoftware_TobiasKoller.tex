% \documentclass[a4paper,12pt,titlepage]{article}
\documentclass[runningheads]{llncs}


%Bibliography
\usepackage[natbib=true, citestyle=apa, bibstyle=apa]{biblatex}
\bibliography{ToBITEnglishCorrectionSoftware.bib}

\usepackage{csquotes}
%\usepackage[utf8]{inputenc}
%\usepackage[english]{babel}
%\usepackage{blindtext}
%\usepackage{microtype}
%\usepackage{graphicx}
%\usepackage{wrapfig}
%\usepackage{enumitem}
%\usepackage{fancyhdr}
%\usepackage{amsmath}
%\usepackage{index}
%\usepackage[onehalfspacing]{setspace}
%\usepackage{nomencl}
%\makenomenclature
%\renewcommand{\nomname}{List of Abbreviations}

%opening

%\makeindex

\begin{document}

%\title{\Large{\textbf{English Correction Software}} }
\title{English Correction Software}
\subtitle{ToBIT Paper}

\author{Tobias Koller}

\institute{University of Applied Science Northwestern Switzerland (FHNW), \\ 4000 Basel, Switzerland}


\maketitle              % typeset the header of the contribution


\begin{abstract}
This paper forms ....

\keywords{Grammarly \and Natural Language Processing \and Language Correction Software}
\end{abstract}


%\let\cleardoublepage\clearpage
%\pagenumbering{roman}

\section{Introduction}
Numerous software solutions on the market promise to help in particular non-native English speakers with grammatical error detection and improving the style and structure of their writing. In the course of the module ``Innovative Topics in Business Information Technology'' (ToBIT), I am going to evaluate different products regarding their functions, ease of use and effectiveness in supporting the writing process. The main goal is to use literature review to discover how effective the correction software is under real conditions.

One widely known and used writing tool for grammar checking, spell checking, and plagiarism is Grammarly®. Since it is one of the leading products in this field and there are innumerable research papers to support, I focus on this particular tool.

After the evaluation of the tools, I will also describe different natural language processing (NLP) techniques that are being applied. In the final part of the document, the findings will be discussed to show clearly the current state of art in this field. A recommendation to University of Applied Sciences and Arts Northwestern Switzerland (FHNW) will be given on what products to consider or how to develop a new product internally.

\pagenumbering{arabic}

\section{Introduction}\label{sec:outl}
The following chapter 

\newpage

\pagenumbering{arabic}

\section{Existing solutions}
\subsection{Grammarly}
Grammarly, a company with its eponymous popular language correction software, aims to improve the communication of people. Max Lytvyn, Alex Shevchenko, and Dmytro Lider founded the company 2009 with a strong focus on supporting the student's writing process. In the meantime, they broadened their scope to businesses as well as personal writers. With Grammarly @edu and Grammarly business, they have their dedicated divisions for those clients. Their correction service is based on analysing the written content in real-time while showing suggestions in the form of correction cards.  With a simple click, the changes can be applied or rejected. To promote learning of the writer further grammatical information regarding the specific issue can be retrieved from the card which can help to decide on whether to accept the changes or not. \citep{grammarly_website}


Just cite \cite{dembsey_closing_2017}
\newline

Text cite \textcite{dembsey_closing_2017}
\newline
paren cite \parencite{dembsey_closing_2017}
\newline

citep paragraph? And so on \citep{dembsey_closing_2017}
The following chapter 
\newline
text quote 
\textquote[{\cite{dembsey_closing_2017}}]{word for word }
\newline
\newpage
\section{Bibliography used}
The list of references is not final and will be extended during the process of writing the ToBIT paper.
\printbibliography
\end{document}
