\documentclass[a4paper,12pt,titlepage]{article}
\usepackage[utf8]{inputenc}
\usepackage[english]{babel}
\usepackage{blindtext}
\usepackage{microtype}
\usepackage{graphicx}
\usepackage{wrapfig}
\usepackage{enumitem}
\usepackage{fancyhdr}
\usepackage{amsmath}
\usepackage{index}
\usepackage[onehalfspacing]{setspace}
\usepackage{nomencl}
\makenomenclature
\renewcommand{\nomname}{List of Abbreviations}

%opening

\makeindex

\begin{document}

\title{\Large{\textbf{English Correction Software}}\\ ToBIT Proposal}
\author{By Tobias Koller \\ tobias.koller@students.fhnw.ch \\ FHNW Basel}

\maketitle
\let\cleardoublepage\clearpage
\pagenumbering{roman}

\section{Abstract}
Numerous software solutions on the market promise to help in particular non-native English speakers with grammatical error detection and improving the style and structure of their writing. In the course of the module ``Innovative Topics in Business Information Technology'' (ToBIT), I am going to evaluate different products regarding their functions, ease of use and effectiveness in supporting the writing process. The main goal is to use literature review to discover how effective the correction software is under real conditions.

One widely known and used writing tool for grammar checking, spell checking, and plagiarism is Grammarly®. Since it is one of the leading products in this field and there are innumerable research papers to support, I focus on this particular tool.

After the evaluation of the tools, I will also describe different natural language processing (NLP) techniques that are being applied. In the final part of the document, the findings will be discussed to show clearly the current state of art in this field. A recommendation to University of Applied Sciences and Arts Northwestern Switzerland (FHNW) will be given on what products to consider or how to develop a new product internally.

\newpage

\nomenclature{ToBIT}{Innovative Topics in Business Information Technology}
\nomenclature{NLP}{Natural language processing}
\nomenclature{FHNW}{Fachhochschule Nordwestschweiz - University of Applied Sciences and Arts Northwestern Switzerland}

\tableofcontents
\newpage


\pagenumbering{arabic}

\section{Outline}\label{sec:outl}
The following chapter gives a rough overview of the structure of the final document. It comprises the three main parts of looking at \textit{existing solutions}, the main part of \textit{assessing the usefulness and effectiveness of numerous products} and the final part of giving more in-depth details about the inner workings of the \textit{NLP techniques} used.

On the leaf level of this tree structure, the supporting research found is mentioned. The full references can be found in the reference list.

\vspace{5mm}
\setlist{nolistsep}
\textbf{English Correction Software}
\begin{itemize}
    \item Introduction
    \item Existing Solutions
    \begin{itemize}
        \item Grammarly
        \begin{itemize}
            \item \cite{dembsey_closing_2017}
            \item \cite{cavaleri_you_2016}
        \end{itemize}

        \item Criterion
        \begin{itemize}
            \item \cite{burstein_toward_2003}
            \item \cite{li_rethinking_2015}
            \item \cite{burstein_automated_2004}
        \end{itemize}

    \end{itemize}
    \item Evaluation of Tools
    \begin{itemize}
        \item Grammarly
        \begin{itemize}
            \item \cite{dembsey_closing_2017}
            \item \cite{cavaleri_you_2016}
            \item \cite{qassemzadeh_impact_2016}
            \item \cite{nova_utilizing_2018}
            \item \cite{ventayen_graduate_2018}
        \end{itemize}

        \item Application in classrooms
        \begin{itemize}
         \item \cite{li_rethinking_2015}
         \item \cite{warschauer_automated_2006}
         \item \cite{grimes_utility_2010}
        \end{itemize}

        \item Audience / context specific
        \begin{itemize}
         \item \cite{li_rethinking_2015}
         \item \cite{patout_towards_2019}
        \end{itemize}
        
        \item Others
        \begin{itemize}
            \item \cite{dodigovic_artificial_2007}
            \item \cite{chen_beyond_2008}
            \item \cite{wang_exploring_2013}
            \item \cite{vojak_new_2011}
            \item \cite{cotos_potential_2011}
            \item \cite{wang_case_2011}
        \end{itemize}

    \end{itemize}
    \item Self-experiment of different tools

    \item Techniques of natural language processing
    \begin{itemize}
     \item Contextual Word Representations
     \begin{itemize}
      \item \cite{bell_context_2019}
     \end{itemize}

     \item Rule Based
     \begin{itemize}
      \item \cite{manchanda_various_2016}
     \end{itemize}

     \item Statistical
     \begin{itemize}
      \item \cite{manchanda_various_2016}
     \end{itemize}
     \item Syntax
     \begin{itemize}
      \item \cite{manchanda_various_2016}
     \end{itemize}
    \end{itemize}
    \item Discussion and recommendation
\end{itemize}
\newpage
\section{Bibliography used}
The list of references is not final and will be extended during the process of writing the ToBIT paper.
\bibliographystyle{apalike}
\bibliography{ToBIT_Bibliography_Proposal}

\printnomenclature 

\end{document}
